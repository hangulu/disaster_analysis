\documentclass[10pt,]{article}
\usepackage{lmodern}
\usepackage{amssymb,amsmath}
\usepackage{ifxetex,ifluatex}
\usepackage{fixltx2e} % provides \textsubscript
\ifnum 0\ifxetex 1\fi\ifluatex 1\fi=0 % if pdftex
  \usepackage[T1]{fontenc}
  \usepackage[utf8]{inputenc}
\else % if luatex or xelatex
  \ifxetex
    \usepackage{mathspec}
  \else
    \usepackage{fontspec}
  \fi
  \defaultfontfeatures{Ligatures=TeX,Scale=MatchLowercase}
    \setmainfont[]{Gotham}
\fi
% use upquote if available, for straight quotes in verbatim environments
\IfFileExists{upquote.sty}{\usepackage{upquote}}{}
% use microtype if available
\IfFileExists{microtype.sty}{%
\usepackage{microtype}
\UseMicrotypeSet[protrusion]{basicmath} % disable protrusion for tt fonts
}{}
\usepackage[margin=1in]{geometry}
\usepackage{hyperref}
\hypersetup{unicode=true,
            pdftitle={Unnatural Disasters: The Data Behind Anthropogenic Climate Change},
            pdfauthor={Hakeem Angulu and Louie Ayre},
            pdfborder={0 0 0},
            breaklinks=true}
\urlstyle{same}  % don't use monospace font for urls
\usepackage{graphicx,grffile}
\makeatletter
\def\maxwidth{\ifdim\Gin@nat@width>\linewidth\linewidth\else\Gin@nat@width\fi}
\def\maxheight{\ifdim\Gin@nat@height>\textheight\textheight\else\Gin@nat@height\fi}
\makeatother
% Scale images if necessary, so that they will not overflow the page
% margins by default, and it is still possible to overwrite the defaults
% using explicit options in \includegraphics[width, height, ...]{}
\setkeys{Gin}{width=\maxwidth,height=\maxheight,keepaspectratio}
\IfFileExists{parskip.sty}{%
\usepackage{parskip}
}{% else
\setlength{\parindent}{0pt}
\setlength{\parskip}{6pt plus 2pt minus 1pt}
}
\setlength{\emergencystretch}{3em}  % prevent overfull lines
\providecommand{\tightlist}{%
  \setlength{\itemsep}{0pt}\setlength{\parskip}{0pt}}
\setcounter{secnumdepth}{0}
% Redefines (sub)paragraphs to behave more like sections
\ifx\paragraph\undefined\else
\let\oldparagraph\paragraph
\renewcommand{\paragraph}[1]{\oldparagraph{#1}\mbox{}}
\fi
\ifx\subparagraph\undefined\else
\let\oldsubparagraph\subparagraph
\renewcommand{\subparagraph}[1]{\oldsubparagraph{#1}\mbox{}}
\fi

%%% Use protect on footnotes to avoid problems with footnotes in titles
\let\rmarkdownfootnote\footnote%
\def\footnote{\protect\rmarkdownfootnote}

%%% Change title format to be more compact
\usepackage{titling}

% Create subtitle command for use in maketitle
\newcommand{\subtitle}[1]{
  \posttitle{
    \begin{center}\large#1\end{center}
    }
}

\setlength{\droptitle}{-2em}

  \title{Unnatural Disasters: The Data Behind Anthropogenic Climate Change}
    \pretitle{\vspace{\droptitle}\centering\huge}
  \posttitle{\par}
    \author{Hakeem Angulu and Louie Ayre}
    \preauthor{\centering\large\emph}
  \postauthor{\par}
      \predate{\centering\large\emph}
  \postdate{\par}
    \date{12/8/2018}


\begin{document}
\maketitle

\subsection{Abstract}\label{abstract}

It is undeniable that the climate of the world is changing. Multiple
studies\footnote{IPCC Fifth Assessment Report: Summary for Policymakers,
  \url{https://www.ipcc.ch/site/assets/uploads/2018/02/AR5_SYR_FINAL_SPM.pdf}}
have pointed out that ``warming of the climate system is unequivocal,
and since the 1950s, many of the observed changes are unprecedented over
decades to millennia. The atmosphere and ocean have warmed, the amounts
of snow and ice have diminished, and sea level has risen.'' This issue
becomes controversial, and often politicized, when we examine the cause
of these rapid and dangerous changes in our climate.

Humans impact the Earth in a number of ways, some of which are
deleterious. While human innovation has leaped forward since the
industrial revolution, so have our reliance on fossil fuels for energy,
greenhouse gas emissions, water pollution, and deforestation. All of
these are intimately connected to the systems that control the planet,
including climate, so it is logical for us to expect that human action
is linked to climate change, and may even be the main cause of it.

This analysis looks at the relationship between anthropogenic emissions
(like CO\(_2\), methane, and other greenhouse gases), anthropogenic
alterations to the landscape (like deforestation, and coral reef
destruction), human energy usage trends (like fossil fuel reliance), and
the rate and intensity of natural disasters. While other manifestations
of climate change, like the disappearance of coastal lands and property,
or the destruction of certain Arctic marine habitats, are important, we
chose to focus on natural disasters because in our opinion, they most
clearly present simultaneously the most dangerous, time-sensitive, and
contemporaneously relevant aspects of this phenomenon.

\subsection{Motivation}\label{motivation}

Both team members are from areas of the world that are currently at high
risk of damage due to increasingly frequent and powerful natural
disasters. Hakeem Angulu is from Jamaica, an island lying in the path of
many hurricanes that originate in the Atlantic (Sandy, Wilma, Gilbert,
etc.), and sometimes in the Gulf Coast (Katrina, Ivan, Nate, etc.).
Louie Ayre is from California, a state wracked by increasingly intense
forest fires (with 3 major blazes in November 2018 alone). For both,
climate change is not only ``real'', but here right now, and affecting
major aspects of their lives. Both understand that it is our duty, as
humans, to attempt to mitigate the effects of this phenomenon, and that
process starts with understanding and recognizing the phenomenon's
origin. This analysis seeks to do that, and tie that conceptual
understanding to concrete social and economic measures to further
bolster their belief that swift action is necessary.

\newpage

\subsection{Hypothesis}\label{hypothesis}

Our hypothesis is as follows:

\textit{There is an association between anthropogenic emissions, landscape alterations, and resource usage trends and the rate and intensity of natural disasters}

Each of the elements of the above hypothesis is well proxied by
predictor and response variables of interest:

\underline{Predictors:}

\begin{itemize}
  \item Anthropogenic emissions:
  \begin{itemize}
    \item CO$_2$ atmospheric levels
    \item Methane atmospheric levels
  \end{itemize}
  \item Anthropogenic landscape alterations:
  \begin{itemize}
    \item Deforestation levels
    \item Coral reef destruction levels
  \end{itemize}
  \item Human resource usage trends:
  \begin{itemize}
    \item Deforestation levels
    \item Coral reef destruction levels
  \end{itemize}
\end{itemize}

\underline{Response:}

\begin{itemize}
  \item The rate and intensity of natural disasters:
  \begin{itemize}
    \item The number of hurricanes making landfall in the United States per year
    \item The average wind speed of a hurricane making landfall in the United States
    \item The amount of land affected by fires in California per year
    \item The numbers of people displaced, injured, and killed due to natural disasters
    \item The amount of flooding in the United States per year
    \item The amount of money spent on repairing post-natural disaster property damage
  \end{itemize}
\end{itemize}

With these variables, we also built a predictivee model to predict
damage induced by natural diasters, a model we hope is useful for
understanding, concretizing, and contextualizing the problem at hand.

\subsection{Data}\label{data}

The data were collected from a multitude of sources, listed below:

\begin{itemize}
  \item National Centers for Environmental Information: National Oceanic and Atmospheric Administration's Storm Events Database\footnote{NOAA, https://www.ncdc.noaa.gov/stormevents/ftp.jsp}.
  
  This database includes information about storms (heavy rain, heavy snowfall, hurricane (typhoon), tropical storm, etc.) in every state in the United States from 1950 to 2018. It includes direct and indirect fatalities, estimated property damage, and estimated crop damage for each storm.
\end{itemize}

\subsection{Methods}\label{methods}

\subsection{Assumptions}\label{assumptions}

\subsection{Results}\label{results}

\subsection{Challenges and
Limitations}\label{challenges-and-limitations}

\subsection{Discussion and Conclusion}\label{discussion-and-conclusion}


\end{document}
